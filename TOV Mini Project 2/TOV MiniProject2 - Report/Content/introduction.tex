This project is based on \texttt{Exercise 4: Crossing the River}. The idea is to help a group of people across a river using a boat; they all start on one side, and have to end up on the opposite side. To reiterate the exercise, the group consists of the following:
\begin{itemize}
\setlength\itemsep{.1em}
	\item 1 mother,
	\item 2 girls,
	\item 1 father,
	\item 2 boys,
	\item 1 police officer,
	\item 1 thief,
\end{itemize}

\noindent and the following contraints must be adhered to:

\begin{itemize}
\setlength\itemsep{.1em}
	\item Max 2 persons on the boat,
	\item Mother not alone with boys,
	\item Father not alone with girls,
	\item Thief not alone with family,
	\item Only the police officer, father and mother can handle the boat.
\end{itemize}

\noindent Trying to solve the exercise manually is not exceedingly difficult; with a little bit of trial and error the group can be moved to the other side, but there is no way of telling if there is a more efficient way, and especially if temporal constraints are added, as in \texttt{Task 2} (\Cref{sec:task2}), it becomes very difficult to guarantee we have found the best (i.e., fastest) solution. We can, however, model the problem in Uppaal, and use that to help us find out if what we found was indeed the fastest solution.