\section{Test Cases}
\subsection{Music Player}
\story{Play .mp3 files.}{High}
{As a user, I, want to listen to the .mp3 files stored on my device.}
{\begin{itemize}
\item I am able to play a .mp3 file stored on the device.
\item I am able to regulate the volume when listening to the song.
\item I am not able to open other file types than .mp3.
\end{itemize}}
{\begin{itemize}
\item Precondition: 
\subitem $-$ The application has launched the MusicPlayer Activity, and is ready to play.

\item Procedure:
\subitem $-$ Press the \texttt{Play} button.
\subitem $-$ Verify music is playing.
\subitem $-$ Press the \texttt{Volume Up} button.
\subitem $-$ Verify volume increased.
\subitem $-$ Press the \texttt{Volume Down} button.
\subitem $-$ Verify volume decreased.

\item Postcondition:
\subitem $-$ Verify all verifications have passed.
\end{itemize}}


\story{Locate .mp3 files on device.}{High}
{As a user, I, want the application to search the directories I specify for .mp3 files.}
{\begin{itemize}
\item When I start the application, it finds the .mp3 files stored in /sdcard/Music.
\end{itemize}}
{\begin{itemize}
\item Precondition: 
\subitem $-$ Application is not running.
\subitem $-$ At least 6 .mp3 files are stored in /sdcard/Music.

\item Procedure:
\subitem $-$ Launch the application.

\item Postcondition:
\subitem $-$ The application launches the MusicPlayer Activity, ready to play.
\end{itemize}}


\story{Obtain beats-per-minute for song.}{High}
{As a user, I, want the application to automatically obtain the beats-per-minute (BPM) of the .mp3 files located.}
{\begin{itemize}
\item Obtain 113 as BPM for Rick Astley's Never Gonna Give You Up, when locating the 12 .mp3 files stored in /sdcard/Music.
\item If no BPM is found for a song, it will not be playable.
\item I am able to manually set BPM for a song.
\end{itemize}}
{\begin{itemize}
\item Precondition: 
\subitem $-$ Application is not running.
\subitem $-$ At least 6 .mp3 files are stored in /sdcard/Music.
\subitem $-$ One of the 6 .mp3 files are Rick Astley's Never Gonna Give You Up.
\subitem $-$ An internet connection is established.

\item Procedure:
\subitem $-$ Launch the application.
\subitem $-$ Locate Never Gonna Give You Up as current song, by pressing the \texttt{Next} button until this happens.

\item Postcondition:
\subitem $-$ Verify the BPM is 113.
\end{itemize}}


\story{Navigate songs.}{High}
{As a user, I, want to be able to skip the current song, listen to the previously played song again. 
Further I want to be able to stop and/or pause the playing song.
It should also be possible to seek in the playing song.}
{\begin{itemize}
\item A \texttt{Next} button must exist, which skips to the next song.
\item A \texttt{Previous} button must exist, which skips to the previously played song.
\item A \texttt{Stop} button must exist, which stops the music.
\item A \texttt{Pause} button must exist, which pauses the music, so it can be resumed from the paused position.
\item A \texttt{SeekBar} must exist, which can seek in the playing song.
\end{itemize}}
{\begin{itemize}
\item Precondition: 
\subitem $-$ The application must be launched in the MusicPlayer Activity.

\item Procedure:
\subitem $-$ Press \texttt{Play} button.
\subitem $-$ Press \texttt{Next} button.
\subitem $-$ Verify the playing song has changed to the next song in queue.
\subitem $-$ Press \texttt{Previous} button.
\subitem $-$ Verify the playing song has changed back to the previous song.
\subitem $-$ Press \texttt{Stop} button.
\subitem $-$ Verify the playing has stopped and the \texttt{SeekBar} has reset.
\subitem $-$ Press \texttt{Play} button.
\subitem $-$ Verify the playing has started with the beginning of song previously playing.
\subitem $-$ Press \texttt{Pause} button.
\subitem $-$ Verify the playing has stopped and the \texttt{SeekBar} is not moving from its position.
\subitem $-$ Press \texttt{Play} button.
\subitem $-$ Verify the playing has started at the paused position of song previously playing.
\subitem $-$ Press the centre of the \texttt{SeekBar}.
\subitem $-$ Verify the playing song has skipped to the chosen position.

\item Postcondition:
\subitem $-$ Verify all verifications have passed.
\end{itemize}}

\story{Keep playing when screen is off.}{High}
{As a user, I, would like to save power by turning off the screen while the music is playing.}
{\begin{itemize}
\item The music keeps playing after the \texttt{Power} button is pressed.
\end{itemize}}
{\begin{itemize}
\item Precondition: 
\subitem $-$ The application must be launched in the MusicPlayer Activity.

\item Procedure:
\subitem $-$ Press \texttt{Play} button.
\subitem $-$ Verify music is playing.
\subitem $-$ Press \texttt{Power} button.
\subitem $-$ Verify screen turns off.

\item Postcondition:
\subitem $-$ Verify the music is still playing.
\end{itemize}}

\story{Adjust music speed.}{Low}
{As a user, I, want the music to match my pace even when the playing song's tempo is not exactly my pace.}
{\begin{itemize}
\item The application must be able to match my pace with appropriate songs, when the smartphone is held in either the hand, the pocket, or on the arm.
\end{itemize}}
{\begin{itemize}
\item Precondition: 
\subitem $-$ The application must be launched in the MusicPlayer Activity.
\subitem $-$ The application must be playing music.
\subitem $-$ The user must have the device on his arm.

\item Procedure:
\subitem $-$ Start running.
\subitem $-$ Increase running pace.

\item Postcondition:
\subitem $-$ Verify the tempo of the playing song increased.
\end{itemize}}

\subsection{Step Counter}
\story{Detect the pace of the user.}{High}
{As a user, I, want the application to detect my pace and display it on the screen. }
{\begin{itemize}
\item Criteria 1
\end{itemize}}
{\begin{itemize}
\item Precondition: 
\subitem $-$ The application must be launched in the MusicPlayer Activity.
\subitem $-$ The user must have the device on his arm.

\item Procedure:
\subitem $-$ Start running.
\subitem $-$ Increase running pace.

\item Postcondition:
\subitem $-$ Verify pace is displayed on the screen and increases.
\end{itemize}}

\subsection{Non-graphical Interface}
\story{As a user, I, would like to save power by turning off the screen while still being able to navigate songs when the screen is off.
A screen is turned off when the power button is pressed.}{High}
{ }
{\begin{itemize}
\item The application can still be controlled after the \texttt{Power} button is pressed.
\end{itemize}}
{\begin{itemize}
\item Precondition: 
\subitem $-$ The application must be launched in the MusicPlayer Activity.

\item Procedure:
\subitem $-$ Press \texttt{Power} button.
\subitem $-$ Verify screen turns off.
\subitem $-$ Perform single tap.
\subitem $-$ Verify application starts playing.
\subitem $-$ Perform double tap.
\subitem $-$ Verify playing changed to next song in queue.
\subitem $-$ Perform triple tap.
\subitem $-$ Verify playing song changed back to previous played song.

\item Postcondition:
\subitem $-$ Verify all verifications have passed.
\end{itemize}}