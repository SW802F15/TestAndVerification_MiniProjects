\section{Test Design}
The success of the entire product is determined based on the completion of these problems: 

\begin{center}
\textit{How can we provide music with an appropriate tempo, compared to the current pace, to the runner through the use of a smartphone?}
\end{center}

\begin{center}
\textit{How can a smartphone application be operated without disrupting the runner's form or concentration?}
\end{center}

For better estimation of completion, we have created multiple user stories, which will serve as basis for the overall assessment of success.

All user stories are given a priority, low or high.
Low priority stories are non-critical for the problems, i.e., the problems can still be satisfied if the story fails. 
High priority stories are critical for the problems, i.e., the problems cannot be satisfied if the story fails. 

For the product to be concluded as \textit{passed} for this acceptance test, all high priority user stories must pass.
However, low priority stories is allowed to \textit{fail} without affecting the overall assessment.

\pagebreak
For a user story to pass, it must adhere to the following conditions, however, some user stories have additional acceptance conditions:
\begin{itemize}
%\item All user stories must be tested with unit tests, describing its functionality. (This is also a part of eXtreme Programming, hence it is done as part of our methodology.)
\item All actions (button press, screen swipe, etc.) must be accommodated within X seconds.
\item The functionality of the user story must work, if activated as prescribed. (Test to pass)
\item All input fields must be boundary tested. (Test to fail)
\end{itemize}

