\section{Test Design}
The success of the entire product is determined based on the completion of this overall user story: 

\begin{center}
\textit{The application must be able to play music, which beats-per-minute matches, with the user's running pace.}
\end{center}

For better estimation of completion, we have created multiple smaller user stories, which will serve as basis for the overall success.

All user stories are given a priority, low or high.
Low priority stories are non-critical for the application, i.e., the application still fulfils the primary user story if they fail. 
High priority stories are critical for the application, i.e., the application cannot fulfil the primary user story if they fail. 

For the product to be concluded as \textit{passed} for this acceptance test, all high priority user stories must pass.
However, low priority stories is allowed to \textit{fail} without affecting the overall assessment.

For a user story to pass, it must adhere to the following conditions:

\begin{itemize}
\item All user stories must be tested with unit tests, describing its functionality. (This is also a part of eXtreme Programming, hence it is done as part of our methodology.)
\item All actions (button press, screen swipe, etc.) must be accommodated within X seconds.
\item The functionality of the user story must work, if activated as prescribed. (Test to pass)
\end{itemize}

